\documentclass{article}
\usepackage[utf8]{inputenc}
% ngerman steht hier für die Neue Deutsche Rechtschreibung
\usepackage[ngerman]{babel}
\usepackage{amsmath}
\usepackage{mathbbol}
\usepackage[backend=biber]{biblatex}
\addbibresource{biblio.bib}

\title{Übung 1.2}
\author{Michael Morandell}
\date{05.10.2023}
\begin{document}
\maketitle
\section*{a)}
% Kommentar
\[
x_{1,2} = \frac{p}{2} \pm \sqrt{\frac{p^2}{4} - q}
\]\cite{grisham2013testament}

\section*{b)}
% Kommentar
\[
x_{1,2} = \frac{-b \pm \sqrt{b^2-4ac}}{2a}
\]

\section*{c)}
% Kommentar
\[
\sum_{k=0}^{\infty}\frac{k^x}{k!} = e^x
\]

\section*{d)}
% Kommentar
\[
\sum_{i=0}^{\infty}(-1)^i \frac{x^{2i}}{(2i)!} = \sin(x)
\]

\section*{e)}
% Kommentar
\[
f(x) = \sum_{n=0}^{\infty}\frac{f^{(n)}(a)}{n!}(x - a)^n
\]

\section*{f)}
% Kommentar
\[
    R \mathbf{v} = 
    \begin{pmatrix}
        \cos\theta & -\sin \theta \\
        \sin \theta & \cos \theta
    \end{pmatrix} \cdot
    \begin{pmatrix}
        x\\y
    \end{pmatrix} =
    \begin{pmatrix}
        x\cos\theta&-&y\sin\theta\\
        x\sin\theta&+&y\cos\theta
    \end{pmatrix}
\]\cite{novik2006his}

\section*{g)}
\[
    \begin{pmatrix}
        1 &2 &\vrule & 3 \\
        4 &5 &\vrule & 6
    \end{pmatrix}
\] 

\section{Benutzen Sie Inline-Math}
\subsection*{a)} Wähle \(n \in \mathbb{N}_0\) beliebig, aber fest. So gilt \(n! = n \cdot (n-1)! \).
\subsection*{b)} Sei \(x\) ein lokales Minimum von \(f\).\\
Dann gilt: \(\exists \epsilon >0 : \forall x' \in \mathbb{B}_{\epsilon} (x) : f(x') \ge f(x) \)

\printbibliography
\end{document}